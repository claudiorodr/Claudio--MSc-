\section{Conclusion}
Conceptualizing, building, and developing an \acrfull{ins} for position estimation was a challenging problem. This work permitted to gain a deeper comprehension, understanding and appreciation of how \acrshort{mems}, and inertial sensors in general, operate. Grasping the multiple levels of complexity and abstraction of decades of knowledge from different fields that this work lies upon was a humbling experience.

Obtaining accurate position knowledge from low-cost inertial sensors is an arduous task that is often considered not possible to achieve without the aid of external sources of positioning or heading information. There was an effort to optimize these sensors to their maximum capacities and prove that a low-cost sensor-based positioning system is possible without the assistance of external sources of positioning. We believe that such was achieved with this work, knowing well our solution has its own limitations and there is further space for improvement. Future work can expand upon this by experimenting with different hardware by utilizing new inertial sensors and microcontrollers, testing new environments such as aerial and nautical, and applying innovative solutions to sensor fusion such as machine learning and deep learning algorithms to learn and better decide how to perform the fusion of sensors.